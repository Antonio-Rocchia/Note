\RequirePackage[l2tabu, orthodox]{nag}

\documentclass
[10pt,        % Font size(10 default)
 a4paper,     % Paper size and format
 %draft,      % Draft mode will improve compilation time but not render figures
 onecolumn,   % Onecolumn or two columns
 fleqn,       % fleqn=left align of math formulas
 %leqno,      % Math formula counter to the left of the formula(right default)
 twoside,     % When printing, oneside=fronte twoside=fronte-retro
 notitlepage, % Does not reserve a page for the title, titlepage does
 openany      % Open chapter anywhere, openright open on right page
]{article}    % article, book, report, letter, beamer

%region Preamble

%region Metadata
\title{Le system call nel sistema Unix}
\author{Antonio Rocchia}
%endregion Metadata

%region Language and encoding settings
\usepackage[utf8]{inputenc}                 % Input encoding for characters written in the .tex document
\usepackage[T1]{fontenc}                    % Output encoding for characters written in the .pdf document
\usepackage[italian, english]{babel}        % Languages used in this file, switch with \selectlanguage{language}, default is italian
%endregion Language and encoding settings

%region Page layout settings
%region Page dimension, margin, header and footer dimensions
\usepackage{geometry}
\geometry{
paper=a4paper,              % Type of paper used
hmargin=2cm,                % Shorthand for left and right margin
bindingoffset=1cm,          % Offset from page binding to the start of the inner margin
top=1cm,                    % Top margin
bottom=0.8cm,               % Bottom margin
includeheadfoot=true,       % Include header and footer in the body size
headsep=1cm,                % Distance from header and main text
headheight=12pt,            % Height of the header
footskip=2cm,               % Distance from the bottom of the foot the end of body
}
%endregion Page dimension, margin, header and footer dimensions

%region Header and footer content
\usepackage{fancyhdr}
\pagestyle{plain}
%endregion Header and footer content

%region Microtype
\usepackage[final,babel=true]{microtype}
%endregion Microtype

%endregion Page layout settings

%region Hyperlinks
\usepackage{hyperref}
\makeatletter
\hypersetup{
colorlinks=true,
linkcolor=blue,
filecolor=magenta,
urlcolor=blue,
pdftitle={\@title},
pdfauthor={\@author}
}
\makeatother
    
\urlstyle{same}
%endregion Hyperlinks

%endregion Preamble

%region modules
%% Code ---------------------------
%%% Dependencies ******************
    \usepackage[dvipsnames]{xcolor} % Used by: listings
%%% ***********************************
\usepackage{listings} % Code display "https://www.ctan.org/pkg/listings"
\lstset{
    backgroundcolor=\color{White},
    basicstyle=\ttfamily,
    identifierstyle=\color{Black},
    stringstyle=\color{Green},
    keywordstyle=\color{NavyBlue},
    commentstyle=\color{Gray},
    extendedchars=true,
    showstringspaces=false,
    % caption
    title=\lstname,
    % frames
    frame=lines,
    breaklines=true,
    postbreak=\mbox{\textcolor{red}{$\hookrightarrow$}\space},
}

\lstnewenvironment{C++}[1][]
    {\lstset{language=C++},#1}
    {}

\lstnewenvironment{C}[1][]
    {\lstset{language=C},#1}
    {}

%%% Snippets """""""""""""""""""""""""
% snippet [C++] "C++ Code block" bA
% \begin{C++}[name=]
% 	
% \end{C++}
% endsnippet

% snippet [C] "C Code block" bA
% \begin{C}[name=]
% 	
% \end{C}
% endsnippet

% snippet code_i "Code Inline" bA
% \lstinline[language=]!"code"!
% endsnippet

% snippet import_code "Import code from a file" bA
% \lstinputlisting[language=,name=]{}
% endsnippet

%% --------------------------------
%endregion modules

%region Body
\begin{document}

%region Title and TOC
\microtypesetup{protrusion=false} % disables protrusion locally in the document
\maketitle          
\tableofcontents % prints Table of Contents
\microtypesetup{protrusion=true} % enables protrusion
%endregion Title and TOC

\section{Le system call di Unix}
Il kernel di Linux espone un API di funzioni che un utente o un processo può chiamare per richiedere che un servizio venga svolto dal sistema operativo a livello kernel.

Ogni linguaggio di programmazione fa uso di queste system call in tantissimi modi. Ad esempio per aprire e creare file, leggere da un file, inizializzare un processo e così via.

Nei linguaggi di programmazione moderni le system call sono esposte al programmatore tramite una serie di funzioni(wrapper) che ne semplificano l'uso ed impediscono che vengano commesse violazioni dell'API.
\section{Lista delle system call Unix}
\begin{itemize}
    \item Controllo dei processi/thread:
        \begin{itemize}
            \item Creazione processi
            \item Terminazione processi
            \item Sospensione processi
            \item Attesa terminazione
            \item Sostituzione di un processo
        \end{itemize}
    \item Gestione del file system
    \item Gestione dei dispositivi
    \item Comunicazione
\end{itemize}

\section{Wrapper delle system call nel linguaggio C}
\subsection{fork()}
fork - Crea un processo figlio
\begin{C}[Includere fork]
    #include <unistd.h>

    int fork();
\end{C}
\paragraph{Descrizione:}
\lstinline[language=C]!fork()! genera un processo figlio duplicando il processo che chiama questa funzione.
Il nuovo processo è chiamato \textit{figlio}, il processo chiamante è chiamato \textit{padre}

Il processo figlio ed il padre \textit{continuano l'esecuzione in spazi di memoria diversi}. Al momento della fork() \textit{il figlio ottiene una copia dello spazio di memoria del padre}
Il padre ed il figlio sono identici eccetto per queste differenze:
\begin{itemize}
    \item Il figlio ha il suo PID unico e diverso dal padre
    \item il figlio non eredita allarmi (\textit{alarm()}) dal padre
    \item Il segnale di terminazione del figlio è sempre \textit{SIGCHLD}
    \item Il figlio eredita le copie dei descrittori(\textit{file descriptor}) aperti dal padre
    \item Il filgio eredita una copia dello stream di direttori aperti dal padre (\textit{opendir()})
\end{itemize}
\paragraph{Valore di ritorno:}
In caso di successo, il processo padre riceve il PID del processo figlio appena generato, e il figlio riceve 0. 
In caso di fallimento, il padre riceve \textit{-1}, non viene generato nessun processo figlio e \textit{errno} viene impostata correttamente.
\paragraph{Esempi:}
Di seguito vengono riportati alcuni esempi di codice che mostrano gli utilizzi tipici di fork(). Si cerca dove possibile di spiegare al meglio cio che succede durante l'esecuzione

\begin{C}[Esempio minimale di chiamata a fork()]
#include <stdio.h>
#include <unistd.h>

int main(void) {
   //  Inizializzo e controllo il valore di ritorno di fork()
   int pid = 0;
   pid = fork(); 
\end{C}
     Il processo figlio ottiene una copia dello spazio di memoria del padre, in questo caso la variabile pid inizializzata a 0.
 Il processo padre ed il processo figlio proseguono l'esecuzione allo stesso momento (in parallelo), ma con una differenza, \textit{fork restituisce il PID del processo figlio generato al padre e restituisce 0 al figlio}.
     Questo valore viene assegnato alla variabile pid.
     Ora possiamo scrivere del codice che permetta al processo di capire se è il padre o il figlio.
\begin{C}
   if(pid==0) { // Sono il figlio
       // Codice del figlio
       printf("Sono il padre");
       return 0;
   } else if(pid > 0) { // Sono il padre
       // Codice del padre
       printf("Sono il padre");
   }
   return 0;
}
\end{C}
Il processo figlio esegue lo stesso codice del processo padre. Pertanto nella maggior parte dei casi è necessario terminare l'esecuzione del figlio prima di uscire dal blocco \textit{if(pid==0)}

%paragraph{Todo: aggiungere più esempi}


\subsection{getpid(),getppid()}
getpid, getppid - restituisce l'identificatore del processo(PID)
\begin{C}[Includere getpid()]
#include <unistd.h>

int getpid();
int getppid();
\end{C}
\paragraph{Descrizione:}
\textit{getpid()} restituisce al processo che chiama la funzione il sui identificatore di processo \textit{(PID)}.

\textit{getppid()} restituisce al processo che chiama la funzione l'identificativo di processo \textit{(PID)} del suo processo padre.
\paragraph{Controllo degli errori}
Questa system call non fallisce mai. Non è necessario alcun controllo.
\paragraph{Esempi:}
Di seguito alcuni esempi di codice che mostrano l'utilizo di getpid() e getppid().

\lstinputlisting[language=C,name=getpid() in azione]{esempi/getpid_base.c}

\subsection{exit()}
exit - Causa la terminazione \textit{volontaria} di un processo.

Un processo può terminare involontariamente, se si tentano azioni illegali o se il processo viene interrotto tramite \textit{segnale} (\textit{kill()}), o volontariamente, se viene eseguita l'ultima istruzione o si esegue la funzione exit();




\end{document}
%endregion Body
